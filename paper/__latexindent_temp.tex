		Our explanatory variable is a state's relationship with the United States. We predict that states with moderate-strength relationship ties that desire improved ties will be more likely to choose costly contributions to signal that desire to the United States. We operationalize this using network statistics from the yearly alliance network. Also because of the potential benefits of signaling reliability to the United States, we compiled data on ideal point estimates relative to the United States in each year. Our primary network statistic of interest is a country's structural equivalence with the United States in the alliance network. This serves as a useful means of capturing the degree to which the country is deeply embedded within or sits on the periphery of the initiating state's security community. As \ref{fig:contr_sequiv} demonstrates, the relationship between structural equivalence and troop contributions follows our expectation reasonably well, with a curvilinear relationship between a country's average contribution and their structural equivalence with the United States.
