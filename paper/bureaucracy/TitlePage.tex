% !BIB program = bibtex

\documentclass[12pt,letterpaper]{article}
\usepackage[utf8]{inputenc}
\usepackage[T1]{fontenc}
\usepackage{amsmath}
%\usepackage{amsfonts}
%\usepackage{amssymb}
\usepackage{makeidx}
\usepackage{graphicx}
\usepackage[normalem]{ulem}
\usepackage{subcaption}
\usepackage{float}
\usepackage{longtable}
\usepackage{multirow}
\usepackage{titlesec}
\setcounter{tocdepth}{2}
\usepackage[margin=1in]{geometry}
\usepackage{ntheorem}
\usepackage{booktabs}
\usepackage{dcolumn}
\usepackage[stable]{footmisc}
\usepackage{setspace}

% font
\usepackage[charter,cal=cmcal]{mathdesign}

\usepackage{ntheorem}
\newtheorem{hyp}{Hypothesis}
\newtheorem{subhyp}{Hypothesis}[hyp]
\renewcommand\thesubhyp{\thehyp.\alph{subhyp}}
\usepackage{tikz}
\usetikzlibrary{arrows, decorations.pathmorphing, backgrounds, fit, positioning, shapes.symbols, chains, decorations.pathreplacing}
\usepackage[round]{natbib}
\bibpunct{(}{)}{;}{a}{}{,~}
\usepackage[space]{grffile}
\graphicspath{{./figures/}}
\usepackage[affil-it]{authblk}
\makeatletter

\def\@maketitle{%
	\newpage
	\null
%	\vskip 2em%
	\begin{center}%
		\let \footnote \thanks
		{\Large\bfseries \@title \par}%
		%\vskip 1.5em%
		{\normalsize
		%	\lineskip .5em%
			\begin{tabular}[t]{c}%
				\@author
			\end{tabular}\par}%
		%\vskip 1em%
		{\normalsize \today}%
	\end{center}%
	\par
	%\vskip 1.5em
	}
\makeatother

\title{Keeping Your Friends Close, But Acquaintances Closer: Why Weakly Allied States Make Loyal Coalition Partners}

\author{J Andr\'{e}s Gannon%
	\thanks{Electronic address: \texttt{jagannon@ucsd.edu} Web: \texttt{jandresgannon.com}}}
\affil{Department of Political Science \\ University of California, San Diego}

\author{Daniel Kent%
	\thanks{Electronic address: \texttt{kent.249@osu.edu} Web: \texttt{dnkent.github.io} \\ We thank members of the Center for Peace and Security Studies (cPASS), in particular Erik Gartzke, Rex Douglass, Matthew Millard, Michael Rubin, Thomas Scherer, and Alexandra Woodruff as well as Jan Angstrom, James Fowler, Luke Sanford, Rachel Schoner, and Camber Warren for their comments and suggestions. Erin Ling, Amanda Madany, Cole Reynolds, Effie Sun, Alexandra Vignau, Erin Werner, and Lisa Yen provided invaluable research assistance. Previous drafts were presented at the 2018 American Political Science Association conference and 2018 Political Networks Conference. This research was sponsored by Office of Naval Research Grant N00014-14-1-0071 and the Department of Defense Minerva Research Initiative. Any opinions, findings, and conclusions or recommendations expressed in this publication are those of the authors and do not necessarily reflect the view of the Office of Naval Research.}}
\affil{Department of Political Science \\ The Ohio State University}

\begin{document}
\maketitle

\begin{abstract}
	Why do states join wartime coalitions despite an absence of a salient national threat or strong ties to the coalition leader? We argue that states make unexpectedly high contributions to coalition warfare as a costly signal of their desire for a stronger relationship with the coalition leader. Conventional explanations for coalition warfare cooperation cannot explain why states without immediate security interests or strong ties to the lead state over-contribute relative to their capacity. Using newly compiled data on relative country-level troop contributions to the war in Afghanistan (2001-2014), we find that states are most likely to contribute a higher share of their armed forces when their current relationship with the US is weaker than one would expect given the consistency of that state's international security interests with that of the United States. Countries that are already closely aligned with the central coalition actors and those with a weak alliance relationship tend to under-commit troops relative to the largest contributors, whose moderate -- but under-performing -- alignments leave substantial room for subsequent gains to be had from signaling their commitment to the leading coalition actor.
\end{abstract}

\end{document}