% !BIB program = bibtex

\documentclass[12pt,letterpaper]{article}
\usepackage[utf8]{inputenc}
\usepackage[T1]{fontenc}
\usepackage{amsmath}
%\usepackage{amsfonts}
%\usepackage{amssymb}
\usepackage{makeidx}
\usepackage{graphicx}
\usepackage[normalem]{ulem}
\usepackage{subcaption}
\usepackage{float}
\usepackage{longtable}
\usepackage{multirow}
\usepackage{titlesec}
\setcounter{tocdepth}{2}
\usepackage[margin=1in]{geometry}
\usepackage{ntheorem}
\usepackage{booktabs}
\usepackage{dcolumn}
\usepackage[stable]{footmisc}
\usepackage{setspace}
\onehalfspacing

% font
\usepackage[charter,cal=cmcal]{mathdesign}

\usepackage{ntheorem}
\newtheorem{hyp}{Hypothesis}
\newtheorem{subhyp}{Hypothesis}[hyp]
\renewcommand\thesubhyp{\thehyp.\alph{subhyp}}
\usepackage{tikz}
\usetikzlibrary{arrows, decorations.pathmorphing, backgrounds, fit, positioning, shapes.symbols, chains, decorations.pathreplacing}
\usepackage[round]{natbib}
\bibpunct{(}{)}{;}{a}{}{,~}
\usepackage[space]{grffile}
\graphicspath{{./figures/}}
\usepackage[affil-it]{authblk}
\makeatletter

\def\@maketitle{%
	\newpage
	\null
%	\vskip 2em%
	\begin{center}%
		\let \footnote \thanks
		{\Large\bfseries \@title \par}%
		%\vskip 1.5em%
		{\normalsize
		%	\lineskip .5em%
			\begin{tabular}[t]{c}%
				\@author
			\end{tabular}\par}%
		%\vskip 1em%
		{\normalsize \today}%
	\end{center}%
	\par
	%\vskip 1.5em
	}
\makeatother

\title{Keeping Your Friends Close, But Acquaintances Closer: Why Weakly Allied States Make Loyal Coalition Partners}

\author{J Andr\'{e}s Gannon%
	\thanks{Electronic address: \texttt{jagannon@ucsd.edu} Web: \texttt{jandresgannon.com}}}
\affil{Department of Political Science \\ University of California, San Diego}

\author{Daniel Kent%
	\thanks{Electronic address: \texttt{kent.249@osu.edu} Web: \texttt{dnkent.github.io} \\ Draft version, please do not circulate. We thank members of the Center for Peace and Security Studies (cPASS), in particular Erik Gartzke, Rex Douglass, Matthew Millard, Michael Rubin, Thomas Scherer, and Alexandra Woodruff as well as Jan Angstrom, James Fowler, Luke Sanford, Rachel Schoner, and Camber Warren for their comments and suggestions. Erin Ling, Amanda Madany, Cole Reynolds, Effie Sun, Alexandra Vignau, Erin Werner, and Lisa Yen provided invaluable research assistance. Previous drafts were presented at the 2018 American Political Science Association conference and 2018 Political Networks Conference. This research was sponsored by Office of Naval Research Grant N00014-14-1-0071 and the Department of Defense Minerva Research Initiative. Any opinions, findings, and conclusions or recommendations expressed in this publication are those of the authors and do not necessarily reflect the view of the Office of Naval Research.}}
\affil{Department of Political Science \\ The Ohio State University}

\begin{document}
	
\begin{singlespace}
\maketitle

\begin{abstract}
We argue that states make unexpectedly high contributions to coalition warfare as a costly signal of their desire for a stronger relationship with the coalition leader. Our theory contributes to discussions about why states cooperate over security affairs and how they shore up that cooperation. Conventional explanations for coalition warfare cooperation focus on the salience of the threat in question or the close relationship between coalition leader and participant, but they cannot explain why states without immediate security interests or strong ties to the lead state over-contribute relative to their capacity. Using newly compiled data on relative country-level troop contributions to the war in Afghanistan (2001-2014), we find that states were most likely to contribute a higher share of their armed forces when their current relationship with the US was weaker than one would expect given each states' international security interests. Countries that are already closely aligned with the central coalition actors and those stranded on the periphery alike tend to under-commit troops relative to the largest contributors, whose moderate alignments leave substantial room for subsequent gains to be had from signaling their commitment to the leading coalition actor.
\end{abstract}
\end{singlespace}

\section{Introduction}
	In 2001 the United States launched the war in Afghanistan with the goal of overthrowing the Taliban and dismantling Al-Qaeda. But it did not do so alone; instead it led a coalition of 50 other states that fought alongside the United States until the war formally ended in 2014.\footnote{This refers to Operation Enduring Freedom, not the ongoing Operation Freedom's Sentinel.} Not every state contributed equally. While the United States sent thousands of troops for the duration of the conflict, others contributed only a handful to fulfill their contractual alliance obligations under NATO. Others that had no formal alliance obligations incurred significant costs by sending troops. New Zealand lost almost a dozen troops in the Afghanistan conflict; a difficult thing for a state leader to justify to their public when the outcome of the conflict is largely immaterial to the state at hand and when it has no alliance obligation to participate.

	This is exemplary of a broader trend in coalition warfare. Some states contribute forces to coalition wars because they care about the material outcome of the conflict and hope to influence that outcome in some significant way. Others contribute forces because alliance obligations or expectations create a cost to free riding. But what motivates some states to pay a heavy cost to fighting alongside another country when traditional security calculations or alliance obligations don't apply? Neither of those theories explain contributions like that of New Zealand to the war in Afghanistan; states that are unaffected by the outcome of the conflict, that have no notable ability to influence the outcome of the war, that experience no reputational cost from refusing to participate, and yet willingly risk high costs from their participation.
	
	Table \ref{table:2001_top} shows that New Zealand is not an isolated incident. Many of the states that contributed the highest share of their armed forces were not those that current theories would anticipate. Of the 10 states that committed the largest share of their military forces to the theater, only Turkey is geographically proximate and only 5 of the 9 non-US countries were NATO members. And although every NATO member participated in at least some capacity, over a half dozen had no physical presence in Afghanistan until more than halfway through the mission.

	\begin{table}[ht]
		\centering
		\begin{tabular}{|lr|}
			\hline
			\textbf{State} & \textbf{Contribution} \\
			\hline
			Denmark & 0.66 \\
			New Zealand & 0.65 \\
			United States & 0.53 \\
			Romania & 0.53 \\
			Netherlands & 0.44 \\
			Germany & 0.44 \\
			Norway & 0.34 \\
			Australia & 0.29 \\
			Turkey & 0.27 \\
			Spain & 0.20 \\
			\hline
		\end{tabular}
	\caption{Top 10 contributors to ISAF (2001) by percent of armed forces. Source: IISS Military Balance (2002). Canada's conflict-leading contributions are omitted because this data only references 2001.}
	\label{table:2001_top}
	\end{table}

	This paper explains such contributions by developing a new theory about coalition warfare participation by states whose primary objective in fighting is not to influence the outcome of the conflict or to maintain their current reputation as allies who reliably fulfill their contractual obligations. We argue that states contribute forces to coalition warfare in order to \textit{develop} their reputation as reliable allies. These states make costly contributions not because they already have close relationships with the coalition leader, but because their relationship is not close enough. As such, their contribution is disproportionately costly to signal their desire for a stronger relationship with their coalition partner. We find that if state military contributions are measured not by the number of troops they contributed, but the proportion of their military forces they contributed, that states seeking to develop a stronger relation with coalition leaders are more likely to have higher proportional troop contributions. In other words, states that wish to increase their ties to central actors in the coalition network seek to do so by over-contributing, sending a larger portion of their armed forces than other states that were formally obligated to partake in coalition operations.
	
	We test our theory using new data on each country's relative contribution to the war effort in each year and each country's latent relationship strength with the United States. Unlike prior analyses that focus on raw troop count, we consider the cost of each state's troop contribution as a ratio of the size of their armed forces. If a small state is trying signal to the coalition leader that it is willing to overexert itself on behalf of that coalition leader, its relative contribution is what matters. Rather than predict a state's participation in war based on the nature of their current relationship, looking at the discrepancy between the closeness of their current relationship and the relationship one would expect given foreign policy interests can help identify the group of states that, like New Zealand, may over-contribute to improve the closeness of the current relationship.
	
	Our analysis finds that %insert findings from test.
	This finding has important implications for understanding the costly means by which states seek to re-align themselves in international alliance networks. Conflict is generally understood as a costly tool which states employ to achieve their international objectives. One of those objectives can be unrelated to the outcome of the conflict itself. States may rely on war efforts to serve as a signal of willingness to undergo large costs in the hopes that this will improve their relationship with central states in the future. This can help explain the ways that states use conflicts which are not of immediate strategic importance to gain the attention and (hopefully) respect of central players in the international system. Simply put, war can be a good excuse for improving ties with important states. Our theory also has broader implications for theories about institutionalization. States are willing to incur costs in an institutional context not when they are already well embedded in those institutions, but when they have the most to gain from further institutionalization.

	This paper proceeds in five parts. In part two we explore existing explanations for coalition warfare, which thus far have focused on the decision to join coalition efforts rather than the extent of their participation. Part three develops our theory by applying the costly signaling theory to coalition warfare through a novel measure of the costliness of a state's contribution to coalition warfare: the relative pressure of that mobilization based on the size of its available armed forces. Part four empirically test this finding by examining coalition contributions during the war in Afghanistan (2001-2014), which presents a ripe test case for our theory given variation in the alliance obligations of the states that participated as well as their level of participation. Section five includes a discussion of the generalizability of our results for the broader theory, explaining how states seek to alter and leverage their position within the network of capable international actors. Section six concludes and discusses areas where we are particularly interested in receiving feedback.

\section{Existing Explanations for Contributions to Coalition Warfare}
	An alliance between two actors constitutes a promise to take a certain action in the event of a particular contingency \citep{altfeld_decisionallytheory_1984}. In the international security context, an alliance is thus often a promise to defend the other actor in the event of a threat to their security \citep{waltz_theoryinternationalpolitics_1979, walt_originsalliance_1987} or to facilitate mutually beneficial cooperation when states have common objectives \citep{keohane_hegemonycooperationdiscord_1984, wolford_showing_2014}. Attempts to explain why countries contribute to alliances to varying degrees can be divided into four different categories; theories of collective action, the balance of threat, alliance dependence, and domestic politics \citep{bennett_burdensharingpersiangulf_1994, haesebrouck_democraticparticipationair_2016}.
	
	The \textit{collective action hypothesis} introduced by \citet{olson_economictheoryalliances_1966} argued that dominant states will end up making the largest contributions because smaller states can simply free ride while continuing to garner the benefits of the alliance relationship writ large. These dominant states, with larger economies and militaries, end up paying a disproportionate burden to secure goods even when the public benefits of those goods are reaped by states that made little contribution themselves. The \textit{balance of threat hypothesis} argues that state contributions should be proportional to the gravity of the threat an issue presents to a state. When faced with a larger threat, a state will contribute more to an alliance coalition that they anticipate mitigating that threat \citep{walt_originsalliance_1987, sandler_natoburdensharing_2014, baltrusaitis_coalitionpoliticsiraq_2010}. The \textit{alliance dependence hypothesis} argues that states in an alliance must balance two competing fears; the fear of abandonment and the fear of entrapment \citep{snyder_securitydilemmaalliance_1984}. Problematically, reducing one of these risks necessitates an increase in the risk of the other. Allied support is thus explained by how a state feels about these relative risks; a states will contribute to an alliance when the fear of abandonment exceeds the fear of entrapment. States that most fear abandonment will be those that are most dependent on the other partner in the alliance meaning the military and economic payoffs of the alliance relationship would be difficult to replace. This theory has been expanded by scholars who note that ``alliance value" explains contributions by states who believe they can leverage their allies partner in their favor \citep{davidson_neoclassicalrealistexplanation_2011}. The \textit{domestic hypotheses} of alliance contributions take multiple forms \citep{,tago_whenaredemocratic_2009, ashraf_politicscoalitionburdensharing_2011, pilster_aredemocraciesbetter_2011, wolford_nationalleaderspolitical_2016}. Theories of state autonomy and domestic society predict that leaders that can insulate themselves from external constraints on their decision-making about alliance contributions are consequently able to do so even when these leaders' preferences regarding contributions differ from those of the public \citep{saideman_ambivalentcoalitiondoing_2016, vonhlatky_ideologyballotsalliances_2018}. Theories of bureaucratic politics instead examine the relationship within the government. Bureaucratic decision-making requires negotiations and bargaining among relevant actors and as such state contributions are the outcome of these bargaining decisions and the environment shaping the bargaining framework \citep{rathbun_partisaninterventionseuropean_2004, mello_democraticparticipationarmed_2014}.

	Other literature on state contributions to joint efforts have examined topics like UN peacekeeping. In this context, state contributions are influenced by network centrality \citep{dorussen_networkedinternationalpolitics_2016}, international factors \citep{mullenbach_decidingkeeppeace_2005}, geostrategic interests \citep{baltrusaitis_friendsindeedcoalition_2008}, and whether they are a democracy \citep{lebovic_unitingpeacedemocracies_2004}. While theoretically important, these findings do not explain the costliness of a state's contribution nor whether that contribution was used as a signal to improve a state's ties with other actors in the network. Their network position is assumed as a static factor explaining the level of contribution when in reality it is a position that states want to actively manipulate. The costliness of UN peacekeeping contributions as a signal is harder to measure considering the risk of casualties and collateral damage is lower and domestic publics are less attuned to their state's participation in UN peacekeeping than active military operations. Although the importance of theorizing differing contributions has been recognized in the context of US-led coalition efforts, this work has focused on single-dyad case studies that would benefit from broader analysis that could confirm the generalizability of their findings \citep{mello_politicsmultinationalmilitary_2018}.

\section{Theory of Joint Warfare as a Costly Signal}
	Why do states cooperate with one another on security issues and how do states shore up and maintain that cooperation? We argue that national defense is a club good, not a pure public good, because the protection benefits of an alliance can be excludable and rival \citep{sandler_clubtheorythirty_1997}. After the end of the Cold War, NATO's security benefits became something the US would leverage via threats of exclusion to encourage risk-sharing and minimize free riding \citep{ringsmose_natoburdensharingredux_2010}. As a result, states within security alliances must avoid the temptation to free ride because doing so could risk the benefits of alliance membership. States thus have to devote resources to maintain the receipt of these club goods and can do so by demonstrating what they bring to the table. For example, President Trump's rhetoric regarding NATO burden-sharing has resulted in conversations within Canada regarding how to avoid the perception that they are free-riding on US security commitments \citep{mckay_whycanadabest_2018}.
	
	Though there is an extensive literature on the ways in which states aggregate their capabilities to bolster their security against foreign threats \citep{waltz_theoryinternationalpolitics_1979}, the security benefits of an alliance do not precisely map onto the security benefits of joining a wartime coalition because coalitions and alliances are not synonymous. Coalitions can be a manifestation of an alliance promise or be an ad hoc relationship oriented toward an immediate goal rather than a broader security arrangement \citep{weitsman_wartimealliancescoalition_2010}. The benefits of the alliance can be reaped even if a participant in the alliance free rides \citep{olson_economictheoryalliances_1966}. During wartime coalitions, however, there are more clear private benefits to states from participating in that coalition. Advocates of the balance of threat perspective point to private incentives actors have to ensure that the aggregate contributions to defense are sufficient to deal with the threat \citep{bennett_friendsneedburden_1997, baltrusaitis_coalitionpoliticsiraq_2010, davidson_neoclassicalrealistexplanation_2011}.
	
	Our theory thus lies at the intersection of research on coalitions and alliances to explain why states contribute more or less to coalitions than we would expect \citep{saideman_ambivalentcoalitiondoing_2016}. We argue that coalitions are used as a costly signal by a state that is dissatisfied with a more general alliance relationship they regard as too weak. States get private good benefits from coalition war-fighting. For some states, that private good is a desirable war outcome. For others, the private good to be gained from coalition war-fighting is improving the quality of your relational tie with the central actors leading that conflict. This tie is most improved when a state's contribution to the coalition effort is a costly signal of their commitment to the conflict which happens when they have over-contributed relative to the contribution that would be expected from a state with their military capacity. Since alliances are a costly signal of one's intention to cooperate, it should also be true that fulfilling alliance-like obligations even when you are not bound by those alliance obligations would also be a signal of your desire to cooperate more in the future \citep{warren_geometrysecuritymodeling_2010}. Participation in war efforts by fellow alliance members is one way states can signal that they are not free riding on the alliance membership and are instead participating in burden sharing because they value the reciprocal and mutually-beneficial alliance commitments.
	
	This private goods theory of coalition conflict, as distinct from alliances, explains why states would vary in the \textit{degree} of their contribution. Acquaintances make a costly contribution when they could have chosen to free ride because their interest is not in the public good of security achieved via victory in the conflict but private interests like a security umbrella down the road or closer economic or diplomatic ties with the state to whom they sent a costly signal of support. Previous literature has noted that you should get a payoff from signaling that you honored an alliance commitment and states that do so get more allies in the future because they are seen as reliable \citep{gibler_costsrenegingreputation_2008}. This is consistent with our logic where you hope that your signal of strongly valuing the relationship causes the other actor to later take actions that recognize that.
	
	\subsection{New Conceptions of Contribution Cost and Dynamic Interstate Alliances}
		We depart from current explanations for state contributions to coalition war efforts in how we conceptualize and empirically identify that contribution. The extent to which a country participates in a coalition war effort is distinct from whether they participate at all. The foundational literature on theories of collective action, the balance of threat, and alliance dependence typically examine whether a country participated while only domestic explanations focus on constraints regarding the extent of contribution \citep{bennett_burdensharingpersiangulf_1994, bogers_missionafghanistanwho_2013}.
	
		Contrasting findings among the first three theories are explained by examining the extent of state contributions rather than just whether they contributed \citep{cranmer_coalitionqualitymultinational_2017}. Previous work examining the extent of state contributions suffers from improper conceptualization. Almost all prior analysis has measured a state's contribution in absolute terms; the number of troops, financial contributions, foreign aid, or peacekeepers sent a particular operation \citep{mello_democraticparticipationarmed_2014, haesebrouck_explainingmemberstates_2016}. This measure makes sense for theories interested in explaining who contribute the most, since absolute terms describe the highest contributors \citep{bogers_missionafghanistanwho_2013}. But absolute contributions does not tell us who contributes more or less than expected nor the relative cost of that contribution for the state in question. It is less costly for Great Britain to send 1,000 forces into a conflict theater than for Poland to do the same since Poland's comparatively small military force makes that mobilization an unusually large undertaking. The former has a substantially larger military force and thus experiences a lower national cost in terms of the burden such a contribution places on its military forces and the zero-sum trade-off of this contribution relative to other security needs. Figure \ref{fig:contrib_map} shows that these two measures -- total troops contributed and percent of troops contributed -- provide different baselines about which states had the most costly contributions to the war in Afghanistan. While the US is by far the largest contributor using the conventional measure of number of troops deployed, looking at the percent of a state's armed forces that were contributed demonstrates that countries like Denmark, New Zealand, Romania, and Australia were devoted to the conflict to a surprising extent. Measuring troops contributed as a ratio of other baselines like population size or GDP yield similar results \citep{bogers_missionafghanistanwho_2013}.

		\begin{figure}[H]
			\centering
			\includegraphics[width=\textwidth]{troops_2001_sidebyside_scaled.png}
			\caption{Country Troop Contributions to war in Afghanistan (2001) Source: IISS Military Balance report}
			\label{fig:contrib_map}
		\end{figure}

		The second problem is how current theories think about alliances. Conventional explanations anticipate the highest contributions coming from states that are dependent on the security guarantee of the central coalition actor and those that perceive a special relationship with the United States \citep{haesebrouck_democraticparticipationair_2016, howorth_securitydefencepolicy_2014, graeger_revivalatlanticismnato_2009, biehl_strategiccultureseurope_2013}. However, this views the alliance relationship as static and oriented towards maintaining continuity in the current relationship, not forward-looking and seeking change. As such, the assumption is that countries that fight together do so because of their closely aligned interests. This misses how fighting together can be a method for \textit{altering} the perceived alignment of interests rather than maintaining them. Using \citet{benson_assessingvariationformal_2016}'s measure of alliance closeness, the relationship does not appear to be positive and linear as current theories would predict \citep{massie_democraticalliesfollowership_2016, haesebrouck_explainingmemberstates_2016, ringsmose_natoburdensharingredux_2010}. Figure \ref{fig:contr_sequiv} shows that the states that contributed the largest share of their armed forces were not the most proximate allies of the US, but rather those with middling ties.

		\begin{figure}[H]
			\centering
			\includegraphics[width=0.55\textwidth]{figures/contributions.png}
			\caption{Troop Ratio by Structural Equivalence with the United States}
			\label{fig:contr_sequiv}
		\end{figure}
		
	\subsection{Benefits of Over-contributing Forces}
		Our theory posits that this curvilinear relationship exists because countries will make costly contributions not when they already have close alliance relations with the United States, but when that contribution can be used to improve an under-performing alliance relationship. The most costly contributions do not come from those with the most well-established ties with the coalition lead \citep{ringsmose_natoburdensharingredux_2010, wolford_showing_2014}, those who care about their broader international reputation \citep{pedersen_bandwagonstatuschanging_2018}, or those who are capitulating to demands from the coalition leader \citep{schweller_newrealistresearch_1997}. Rather, over-contributing is a way to send a separating signal that even though a state may be less closely aligned with the central actor prior to or at the outset of the conflict, it is willing to incur costs for the sake of the central actor.

		Costly over-contributions should serve an alliance signaling function under two conditions. First, the over-contributing state must be dissatisfied with the current state of the alliance relationship. While the conventional alliance hypotheses argues that the highest contributions will come from states that have a special relationship with the United States or that are currently dependent on the security guarantee, we argue instead that states that want a more tightly linked relationship with the United States will have more costly contributions. States that already have a special relationship or a reliable security guarantee do not need to over-contribute because their relationship is, in some sense, locked-in. These states have some flexibility in their expected contributions and are unlikely to lose their special relationship status simply because they did not go above and beyond in contributing to the coalition effort. They just need to contribute enough that they are not seen as free riders. This identification of ``latent alliance strength" allows us to differentiate predictable alliance ties from those that are unanticipated. Given the close similarity in foreign policy preferences, the deep alliance relationship between the US and the UK is not surprising. The non-existence of any US-Venezuela security alliance is similarly predictable if a hunch was made based on the non-alignment of those states' foreign policy preferences. We would not expect either the UK or Venezuela to be dissatisfied with the current state of their security alliance with the US even though one represents a close security alliance and the other a non-existent one. Our intuition is motivated by states that don't fit this category like Australia who -- despite historically consistent interests with the US regarding intelligence and security threats -- has a weak, informal, and largely ad-hoc security relationship \citep{fruhling_anzusreallyalliance_2018}.
		
		Importantly, our theory is not necessarily inconsistent with theories of domestic politics since domestic factors are a potential reason a state would want to establish itself as a closer ally with a powerful country \citep{tago_whenaredemocratic_2009, pilster_aredemocraciesbetter_2011, wolford_nationalleaderspolitical_2016}. The domestic origins of alliance-motivated coalition contributions have been witnessed in Egypt in the 1960's \citep{barnett_domesticsourcesalliances_1991}, the United Kingdom during the Iraq War \citep{davidson_americaallieswar_2011}, and Canada since the war in Afghanistan \citep{massie_alliancevaluestatus_2018, mckay_whycanadabest_2018}. In these cases, political leaders who fear domestic opposition from other government actors like bureaucrats or non-partisans or from non-government actors like voters or interest groups may use alliances with other states to demonstrate competent policymaking.
		
		Second, the leading state must perceive the contribution as an effort undertaken for the leading state's benefit. For an over-contribution to be a costly signal of one's commitment to their ally, the private incentives to contribute cannot be egoistic security \citep{davidson_americaallieswar_2011}. If the contributing partner garners private security benefits from participating in the war effort, the central coalition actor would be less likely to interpret the over-contribution as a signal of good will and would instead interpret it as a self-interested intervention that would have occurred irrespective of its benefits to the coalition leader \citep{tago_whystatesjoin_2007, lake_hierarchyinternationalrelations_2009, chapman_securingapprovaldomestic_2011}. Because the value of the signal is determined by its cost, not by its effect, it does not matter if the signaling state sends forces that can materially influence the outcome of the conflict \citep{davidson_americaallieswar_2011}. It only matters that the signal is interpreted as one that was costly for the sender irrespective of its effects. In most cases, the powerful state not only does not need support from smaller state in order to win the conflict, but may actually want to maintain control of the forces that will determine the conflict outcome. This is consistent with alliance theories about the autonomy benefits one actor gets from asymmetric alliances \citep{morrow_alliancesasymmetryalternative_1991}. As such, smaller states do not want to make contributions that are too influential in the conflict's outcome because of the risk of intruding upon the powerful state's decision-making. Importantly, this differs from other perspectives like \citet{bennett_burdensharingpersiangulf_1994} since the US is not exercising leverage over smaller states to induce their contributions, instead the momentum comes from the smaller states whose contribution does more to serve their interests than the interests of the powerful state.

		The cost of partner contributions to coalition warfare is salient in the minds of policymakers and noticed by coalition leaders \citep{ringsmose_natoburdensharingredux_2010}. US Secretary of Defense Gates noted during the war in Afghanistan that there was a ``two-tiered alliance" that made a distinction between ``allies willing to fight and die to protect people's security and others who are not". This has been observed with Nordic states who make costly contributions to improve their relationship with the US because that relationship improves their international status \citep{pedersen_bandwagonstatuschanging_2018}. Our theory differs importantly from the ``alliance value" explanation provided by \citet{davidson_americaallieswar_2011} and \citet{massie_democraticalliesfollowership_2016} who hypothesize that it's about present alliance value. We argue instead that states with a high, but stable current alliance value have less incentive to over-contribute because they don't get any additional payoff from doing so and there is a lower risk of fallout from making a normal contribution since their ties are already close \citep{davidson_headingexitsdemocratic_2014}. This is why small states are able to free ride and take the relationship for granted as pointed out by \citet{keohane_biginfluencesmall_1971}, but we don't always see that because those that desired an improved tie won't free ride and will in fact try to signal that they are not free riders. Canada, for example, was concerned about accusations of free riding on the US alliance and thus intensified their participation in Afghanistan to elevate their status in the eyes of American policymakers \citep{massie_alliancevaluestatus_2018}. If the US-Canada relationship had already been sufficiently strong, there would be a reduced need to signal your value for the relationship. Rather, acquaintances that would benefit from signaling a change in their desired alignment tie will undertake more intense participation \citep{gartzke_contractsfriendsalliances_2012, gibler_priorcommitmentscompatible_2004}.

		\begin{hyp}
			A state with an interest in improving its security alliance with the United States will make more costly contributions than a state that is satisfied with its current security alliance.
		\end{hyp}

		Our theory, presented in figure \ref{fig:theory}, thus predicts a non-linear relationship between the current strength of a state's alliance with the central coalition actor and that state's expected cost of contributing to the coalition war effort. The closest allies of the central coalition actor may make the highest absolute contributions by virtue of their size and military strength, but they are not the states that make the highest relative contributions since they have no incentive to incur the costs that high relative contributions entail. Instead, it's those that want to be closer allies, relatively speaking, that contribute the most. This more closely aligns with the simple observation in figure \ref{fig:contr_sequiv} that shows a curvilinear relationship.

		\begin{figure}[H]
			\centering
			\begin{tikzpicture}
			% Graph axes
			\draw[thick,<->] (0,5) node[above]{Expected Cost of Contribution} -- (0,0) -- (8,0) node[right]{Relationship Tie Strength};

			% Contribution Curve
			\draw[thick] (0,0)..controls (5,5) ..(8,2.5);

			% Desired Relationship Tie
			\draw[dashed] (4,0) -- (4,3.5);
			\draw[dashed] (6.5,0) -- (6.5,3.5);
			\draw[decorate,decoration={brace,mirror,raise=5pt}, thick]
			(4,0) -- (6.5,0);
			\node at (5.25,-0.7) {Desire Improved Tie};
			\end{tikzpicture}
			\caption{Theory of State Contributions to Wartime Coalitions}
			\label{fig:theory}
		\end{figure}

\section{Research Design}
	\subsection{Data}
		Our dependent variable is whether a state contributed troops as a costly signal of their desire for a stronger relationship with the United States. We measure this using data on country-year troops contributions to the war in Afghanistan. Rather than use a raw measure the number of troops a country sent to Afghanistan, we measure the cost of that contribution as the share of a state's armed forces that were deployed for the war. Data on each country's annual troop contributions to the war as well as their overall military personnel were collected from the International Institute for Strategic Studies (IISS) Military Balance reports from 2002 -- 2015 \citep{internationalinstituteforstrategicstudies_militarybalance_2002}.\footnote{The reports publish data on the prior year's activities, so all annual reports have been lagged so they reflect the calendar year they describe rather than the year of publication.} Military personnel and deployment data from IISS has been used widely by previous scholars \citep{walter_buildingreputationwhy_2006, rovner_hegemonyforceposture_2014, beckley_emergingmilitarybalance_2017, henke_politicsdiplomacyhow_2017} and unlike other sources used for the war in Afghanistan, benefits from coverage over the course of the entire war. Figure \ref{fig:afghan_total} shows the total number of troops in Afghanistan during the initial years of the war and figure \ref{fig:troop_hist} shows the distribution across states. Some states like Denmark, New Zealand, the United States and Romania contributed roughly 3 times the size of their armed forces than did the median contributor. 

		\begin{figure}[H]
			\centering
			\includegraphics[width=0.5\textwidth]{figures/country_troop_boxplot.png}
			\caption{Total Foreign Troops in Afghanistan (2001-2005) Source: IISS Military Balance Report}
			\label{fig:afghan_total}
		\end{figure}
		
		There are a variety of ways to measure burden sharing including the number of troops as a share of population size, armed forces size, armed forces in the coalition, and share of GDP \citep{hartley_natoburdensharingfuture_1999}. The costliness of a state's contribution is measured using troop ratios as opposed to financial cost since the risk of casualties and collateral damage are two costs unique to personnel contributions that states are attentive to when deciding whether and to what extent they should participate in coalition warfare \citep{ringsmose_natoburdensharingredux_2010, chivvis_topplingqaddafilibya_2014, haesebrouck_natoburdensharing_2017}. This measure thus best captures the security burden a country has willingly undertaken to fight in another state's war. We do not distinguish between the type of troop or whether they were in an active combat zone because of unreliability of that data on the annual level \citep{bogers_missionafghanistanwho_2013}. Furthermore, while military tacticians may interpret troops sent to a combat zone as a more costly contribution than those sent elsewhere, national politicians -- the intended recipients of that signal -- are less likely to meaningfully make that distinction.

		To avoid the statistical complications of modeling a curvilinear relationship, we reconceptualize the independent variable as a function of ``latent alliance strength" rather than observed alliance strength. For any dyad, the latent alliance strength is the difference between the expected alliance strength given the alignment of security interests between the two states and the observed alliance strength. To operationalize ``latent alliance strength", we use data on UN voting patterns during the war in Afghanistan as a measure of security interest alignment \citep{bailey_estimatingdynamicstate_2017}. Only votes on security issues are considered and the votes are lagged by one year to avoid confounding the relationship or accounting for votes that occurred after troops were deployed in a given calendar year. For example, UN voting data from the US and Canada in 2005 is not considered for Canada's troop deployment in 2005 since those votes could have occurred after the troops were deployed. Observed alliance strength is calculated using previous measures of the scope and depth of formal military alliances \citep{benson_assessingvariationformal_2016}. This is preferable to binary alliance measures since it captures variation in the tightness of a particular alliance tie as well as changes in the nature of a particular alliance over time.
		
		We then estimate the alliance strength two states should have given how closely aligned their security interests seem to be based on the UN voting data. This is consistent with operationalizations used in previous work on coalition warfare \citep{wolford_politicsmilitarycoalitions_2015}. We assume that states with similar security interests should have a formal alliance relationship that reflects this alignment. Of course, that is not always the case. Our ``latent alliance strength" variable is thus the difference between predicted alliance strength and actual alliance strength. We predict that a state whose interests are very closely aligned with the US but that has only a middling relationship with the US should behave as current theories predict closely aligned state will behave (i.e. they will make a costly contribution to the coalition war effort). They will do so in order to signal their desire for a stronger relationship that better represents their true alignment of interests than the current alliance relationship. The earlier example comparing the UK, Venezuela, and Australia is illustrative of this concept. Australia represents an example of a state with a high latent relationship -- meaning an alliance relationship that should be stronger than what one observes. We predict that states that have a weaker relationship with the US than they desire are more likely to choose costly contributions to signal that dissatisfaction to the United States in the hopes that such a signal will curry favor and result in an improved relationship. States that are satisfied with their current relationship with the US (UK and Venezuela), regardless of the strength of the current relationship, are less likely to make unexpectedly costly contributions.

		\begin{figure}[ht]
			\centering
			\includegraphics[width=0.5\textwidth]{figures/troops_hist_largebin.png}
			\caption{Distribution of Country Contributions to Afghan war (2001-2005) Source: IISS Military Balance report}
			\label{fig:troop_hist}
		\end{figure}
		
		As figure \ref{fig:contr_sequiv} demonstrated, the relationship between latent alliance strength and troop contributions follows our expectation reasonably well, with a curvilinear relationship between a country's average contribution over the course of the war and their latent alliance strength with the United States. There are some notable outliers to this trend. Canada, for example, contributed the largest share of its armed forces to the war after 2002 yet has a low latent alliance strength measure since its formal alliance tie with the US is as strong as the voting data predicts. Prior research on Canada's contribution to ISAF indicates this outlier is still consistent with our theory and not a cause for concern. There was a consensus among elites about the need to boost Canada's reputation as a reliable and valued ally, so they contributed more troops than they would have otherwise \citep{massie_alliancevaluestatus_2018}.
		
	\subsection{Model}
		Our unit of analysis for these models is the country-year, spanning from 2001-2005. Though we include and are primarily interested in measures of network statistics, because our outcome of interest is a node-level attribute, we employ generalized linear models and generalized additive models (GAMs) as opposed to methods for modeling tie formation (e.g. exponential random graph and latent space models).\footnote{We are also considering extensions of network models for node formation, i.e. \citet{fosdick_testingmodelingdependencies_2015}, at least as a robustness check of sorts. The only issue with the network approach being that said models are not designed to have the same effect flexibility as GAMs.} For each dependent variable, we estimate a baseline model focused only on effects of interest and then a second which includes controls for distance and democracy. In the former, we first fit a GLM with a polynomial term. But because a polynomial term forces a specific curvilinear relationship upon the data, we also fit a generalized additive model with smoothing terms to see if a similar relationship emerges under a less restrictive format.

		\subsubsection{Model 1: Initial Decision to Commit Troops}
			The curvilinear relationship in figure \ref{fig:contr_sequiv} largely results from the data likely capturing two separate processes. First, capturing why the data is zero-inflated, states initially decide whether or not they will contribute any troops. In this section we model that choice, leaving the amount of troops provided for the next part. Our model for sending troops is represented by equation (1), where `equiv' represents structural equivalence in the alliance network with the United States\footnote{There are multiple possible measures of structural equivalence. We draw upon the Pearson coefficient, which compares the number of actual shared numbers to that one would expect in a truly random network.}. A state's distance from the U.S. in terms of its ideal point estimate is captured by `ideal'. \citep{bailey_twodimensionalanalysisseventy_2018} Last we include terms controlling for democracy and the state's distance from Afghanistan.

			%% Note I've dropped centrality (density) because it is multicollinear with structural equivalence
			\vspace{-2em}
			\begin{equation}
			\text{Contribution}_{it} = \beta_0 + \beta_1\text{equiv}_{it} +  \beta_2\text{ideology}_{it} + \beta_3\text{dem}_{it} + \beta_4\text{dist}_{it}
			\end{equation}
	
		\subsubsection{Model 2: Decision to Over-commit Troops}
			Next, we subset the data to only look at contributors and then examine the predictors of how many troops a country commits, relative to its available number of troops. Here our equations are almost the same, but we include a squared employ linear, as opposed to logistic, regressions because the dependent variable is continuous. With respect to the linear model, we avoid polynomial terms because they force the specified relationship onto the data, instead testing for the expected positive relationship if a linear model is fit.\footnote{Note that we expect strong ties to commit more than weak ties, meaning an expected positive slope.} After confirming these results we switch to a generalized additive model to explicitly capture non-linear effects.

			\vspace{-2em}
			\begin{equation*}
				\text{troop\_ratio}_{it} = \beta_0 + \beta_1\text{equiv}_{it} +
				\beta_2\text{ideology}_{it} + \beta_3\text{dem}_{it} + \beta_4\text{dist}_{it}
			\end{equation*}

\section{Results}
	With respect to a state's decision to commit any troops to Afghanistan, we see that estimates for structural equivalence and ideal point distance are robust to specification and move in the expected directions. In particular, when it comes to the decision to commit troops at all, we see that ideological similarity with the United States is particularly powerful as a predictor. Figure \ref{fig:logit} demonstrates that the relationship is quite clear, with almost all contributors sitting close to the United States ideologically. Moreover, the predicted probabilities, which we derive from our logistic regression estimates, map on relatively cleanly to the actual data. However, it is important to note that our theory is focused on the \textit{amount} of troops contributed, meaning the curvilinear relationship corresponds to the next set of models. However, it is important to note that the cluster of states which contribute are not those perfectly aligned with the US, but those closely aligned -- meaning they have room to gain from subsequent cooperation and, presumably, grow closer, as our theory expects.

	\begin{figure}[H]
		\centering
		\includegraphics[width=0.725\textwidth]{logit_coef.png}
		\caption{Model 1: Decision to commit any troops. FE corresponds to fixed effects.}
		\label{fig:logit_reg}
	\end{figure}
	
	Now that we are looking solely at the size of a state's contributions, we see that structural equivalence emerges clearly as the primary predictor. Though some of the largest contributors, such as New Zealand, sit on the periphery of the alliance network, the more deeply a state is embedded in the United States' security community, then the greater its expected contribution, on average. This fits generally with our expectation, where, even though the relationship is not perfectly linear, we expect states that are closely aligned with the initiating state to contribute more than than those weakly aligned.

	Interestingly, democracy is associated with a lower troop contribution than non-democracies. This actually aligns with work that has argued that that democracies are less reliable allies and more likely to pull out of coalition conflicts early \citep{massie_whydemocraticallies_2016, gartzke_whydemocraciesmay_2004}. While a slightly different context than our theory, these arguments provide a logical extension as to why we should expect that democracies may actually be less likely to over-contribute.
	
	\begin{figure}[H]
		\centering
		\includegraphics[width=0.65\textwidth]{figures/logit.png}
		\caption{Predicted probability of committing troops against actual decision}
		\label{fig:logit}
	\end{figure}

	\begin{figure}[H]
		\centering
		\includegraphics[width=0.725\textwidth]{figures/linear_coef.png}
		\caption{Model 2: Number of troops contributed, relative to number of available troops}
		\label{fig:linear_reg}
	\end{figure}
	
	Moving forward, we estimate a series of GAMs to test our core curvilinear hypothesis. GAMs are similar in structure to GLMs, but model relationships as a result of functions (which can be non-linear) rather than a constant coefficient. This generally takes the form of:
	
	\begin{equation*}
	f(x) = y_i = \alpha + f_1(x_{i1}) + f_2(x_{i2}) + ... + f_p(x_{ip}) + \varepsilon_i
	\end{equation*}
	
	where each function, $f_1, f_2, f_3, ..., f_p$ models the relationship between predictor and outcome, but can easily accommodate linear and non-linear effects.
	
	\begin{figure}[ht]
		\centering
		\begin{minipage}{.5\textwidth}
			\centering
			\includegraphics[width = \linewidth]{figures/no_can_gam.pdf}
			\captionof{figure}{GAM without Canada}
			\label{fig:test1}
		\end{minipage}%
		\begin{minipage}{.5\textwidth}
			\centering
			\includegraphics[width= \linewidth]{figures/can_gam.pdf}
			\captionof{figure}{GAM including Canada}
			\label{fig:test2}
		\end{minipage}
	\end{figure}
	
	Across both models -- with and without Canada -- we find that smoothing function is statistically significant and the function corresponds well to the hypothesis, with a non-linear curve corresponding to moderately structurally equivalent states.

\section{Implications}
	Because alliance membership is a club good, not a pure public good, worries about being excluded from the security benefits of an alliance can incentivize a state to try to shore up the alliance by sending a costly signal of their commitment to their alliance partners. ``Today, we should expect the European allies to find that the best way to strengthen (or avoid weakening) their bonds with the United States is to contribute to out-of-area operations like ISAF. According to this line of reasoning, the allies who put the highest premium on NATO’s traditional products should be the ones – together with the United States – shouldering the heaviest burdens in Afghanistan" \citep{ringsmose_natoburdensharingredux_2010}.

	Our findings runs counter to that expected by theories about the balance of threats. For the balance of threat hypothesis, the states with the highest contributions are those that are most concerned about the threat the opposing state presents to their security \citep{haesebrouck_democraticparticipationair_2016}. Yet, when contributions are measured as internal costs relative to the contribution a state could have made, the highest contributions are from states that appear to be opting into a war in which they have no material stake. Denmark, New Zealand, and Romania, 3 of the top 4 contributors to ISAF in 2001 by our measure, are not the states that are most concerned about the threat that the Taliban and Al Qaeda posed to their national security. This empirical observation is not explained by the theory that states enter an alliance in order to balance against threats \citep{walt_originsalliance_1987}.

	Our findings also contrast those who have tested domestic politics explanations in the context of the war in Afghanistan. NATO's formal institutional structure was supposed to render public opinion less relevant as an explanation for coalition participation \citep{kreps_eliteconsensusdeterminant_2010}. However, we find that while NATO membership and its accompanying obligations to avoid accusations of defection may be sufficient to explain why states do not free ride, that does not explain why states like New Zealand that are not bound by alliance obligations end up over-contributing.

	Previous theories described under the rubric of the alliance hypothesis also fail to explain this findings. For previous theories, ally support is expected when you can leverage your allies' power in your favor and is thus a measure of closeness \citep{davidson_neoclassicalrealistexplanation_2011}. This expects states that are the closest allies with the United States to be the largest contributors to US war efforts. Yet the relationship we see appears non-linear. These states do not over-contribute because the United States is their best protector or because they think they have the ability to influence the United States through their international clout as \citet{ringsmose_natoburdensharingredux_2010} finds. Rather, it is the states that want the United States to be their best protector or that \textit{want} the ability to influence international relations with the United States that are most likely to over-contribute. It's also not the case that the US coerced its closest allies into participating as argued by \citet{kupchan_natopersiangulf_1988} because those are not the states that ended up doing the most participating relative to what they could have contributed. Thus, desire for stronger ties and the expectation that over-contribution will positively signal that desire explains our findings in a way that differs from previous theories.

	While this analysis is limited to an examination of the war in Afghanistan, there is suggestive evidence that the theory holds for other coalition conflicts. During the war in Libya, Denmark make a concerted effort to over-contributed forces, particularly air forces to the bombing campaign, in order to demonstrates their ``relevance and trustworthiness to its great power allies in NATO, especially the United States" \citep{jakobsen_prestigeseekingsmallstates_2018, dicke_natoburdensharinglibya_2013}. This effort appears to have paid off. US Secretary of Defense Robert Gates commended Denmark for its costly contribution to the conflict when he publicly noted that Denmark ``...with their constrained resources, found ways to do the training, buy the equipment, and field the platforms necessary to make a credible military contribution."
	
	As such, these calculations alter conflict outcomes because the quality of coalitions impacts factors like military skill, coordination, and legitimacy \citep{auerswald_natoafghanistanfighting_2014, saideman_ambivalentcoalitiondoing_2016, cranmer_coalitionqualitymultinational_2017}. But having more eager partners is not always an obvious gain. Receiving coalitional support from acquaintances rather than close allies could reduce the ease of coordination and increase the cost of side payments \citep{papayoanou_intraalliancebargainingbosnia_1997, morrow_alliancesasymmetryalternative_1991, wolford_politicsmilitarycoalitions_2015}. During the Korean War, for example, the United States turned down Pakistan's offer of coalition forces because of fears it reduce US control over the conflict and cause operational inefficiency \citep{stueck_koreanwarinternational_1997}. This points to the importance of understanding who joins military coalitions and why \citep{wolford_politicsmilitarycoalitions_2015}. 

\section{Conclusion}
	A state's closest allies will not be the states most invested in fighting alongside that state. Rather, weakly aligned states that desire a closer relationship are most likely to overcontribute forces to coalition warfare. This finding produces a different list of contributors than what current theories predict.
	
	The study of state contributions to coalition warfare provides unique insights if those contributions are measured relative to the cost they impose onto the contributor rather than the utility they provide regarding the conflict outcome. Not every state contributes to a war effort in order to influence the outcome of the war as theories of balance of threats predict. But it is also not the case, as predicted by the collective action hypothesis do states simply free ride when the conflict outcome is immaterial or, as predicted by the alliance dependence hypothesis, that the stronger your alliance, the more you contribute.

	Rather, because war is costly, it can serve as a signaling function for states that want to convey their desire to improve their relationship with other states in the international system. By accepting comparatively large costs to fighting alongside another state, especially when such fighting does not otherwise benefit you, states can signal how much they value an alliance relationship in a way that they anticipate generating reciprocity and payoffs down the road. Future work should examine whether states are making a smart bet in anticipating future payoffs from costly over-contributions; it is possible their gamble is incorrect. But for now, it suffices to say that states with the most interest in future payoffs from a better alliance relationship are the same states most likely to separate themselves from the rest of the coalition pact by over-contributing to wartime coalitions in the hopes that doing so signals their reliability.

\bibliographystyle{apsr}
\bibliography{isaf_alliances}

\end{document}