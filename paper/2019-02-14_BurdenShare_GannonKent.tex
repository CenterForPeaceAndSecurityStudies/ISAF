% !BIB program = bibtex

\documentclass[12pt,letterpaper]{article}
\usepackage[utf8]{inputenc}
\usepackage[T1]{fontenc}
\usepackage{amsmath}
%\usepackage{amsfonts}
%\usepackage{amssymb}
\usepackage{makeidx}
\usepackage{graphicx}
\usepackage[normalem]{ulem}
\usepackage{subcaption}
\usepackage{float}
\usepackage{longtable}
\usepackage{multirow}
\usepackage{titlesec}
\setcounter{tocdepth}{2}
\usepackage[margin=1in]{geometry}
\usepackage{ntheorem}
\usepackage{booktabs}
\usepackage{dcolumn}
\usepackage[stable]{footmisc}
\usepackage{setspace}
%\linespread{1.25} 
\onehalfspacing
% font
\usepackage[charter,cal=cmcal]{mathdesign}

\usepackage{ntheorem}
\newtheorem{hyp}{Hypothesis}
\newtheorem{subhyp}{Hypothesis}[hyp]
\renewcommand\thesubhyp{\thehyp.\alph{subhyp}}
\usepackage{tikz}
\usetikzlibrary{arrows, decorations.pathmorphing, backgrounds, fit, positioning, shapes.symbols, chains, decorations.pathreplacing}
\usepackage[round]{natbib}
\bibpunct{(}{)}{;}{a}{}{,~}
\usepackage[space]{grffile}
\graphicspath{{./figures/}}
\usepackage[affil-it]{authblk}
\makeatletter

\def\@maketitle{%
	\newpage
	\null
%	\vskip 2em%
	\begin{center}%
		\let \footnote \thanks
		{\Large\bfseries \@title \par}%
		\vskip 1em%
		{\normalsize \today}%
	\end{center}%
	\par
	%\vskip 1.5em
	}
\makeatother

\title{Keeping Your Friends Close, But Acquaintances Closer: Why Weakly Allied States Make Loyal Coalition Partners}

\begin{document}
	
\begin{singlespace}
\maketitle

\begin{abstract}
Why do states join wartime coalitions despite an absence of a salient national threat or strong ties to the coalition leader? We argue that states make unexpectedly high contributions to coalition warfare as a costly signal of their desire for a stronger relationship with the coalition leader. Conventional explanations for coalition warfare cooperation cannot explain why states without immediate security interests or strong ties to the lead state over-contribute relative to their capacity. Using newly compiled data on relative country-level troop contributions to the war in Afghanistan (2001-2014), we find that states are most likely to contribute a higher share of their armed forces when their current relationship with the US is weaker than one would expect given the consistency of that state's international security interests with that of the United States. Countries that are already closely aligned with the central coalition actors and those with a weak alliance relationship tend to under-commit troops relative to the largest contributors, whose moderate -- but under-performing -- alignments leave substantial room for subsequent gains to be had from signaling their commitment to the leading coalition actor.
\end{abstract}
\end{singlespace}

\section{Introduction}
	In 2001 the United States launched the war in Afghanistan with the goal of overthrowing the Taliban and dismantling Al-Qaeda. But it did not do so alone; instead it led a coalition of 50 other states that fought alongside the United States until the war formally ended in 2014.\footnote{This refers to Operation Enduring Freedom, not the ongoing Operation Freedom's Sentinel.} Every state's contribution differed. While the United States sent thousands of troops for the duration of the conflict, some European states contributed only a handful to fulfill their contractual alliance obligations under NATO. Other participants that had no formal alliance obligations incurred significant costs by sending troops. New Zealand, for example, lost almost a dozen troops in the Afghanistan conflict -- a difficult thing for a state leader to justify to their public when the outcome of the conflict is largely immaterial to the state at hand and when it has no contractual obligation to participate.

	This is exemplary of a broader trend in coalition warfare. Some states contribute forces to coalition wars because they care about the material outcome of the conflict and hope to influence that outcome in some significant way. Others contribute forces because alliance obligations or expectations create a cost to free riding. But what motivates some states to pay a heavy cost to fighting alongside another country when traditional security calculations or alliance obligations don't apply? Neither of those theories explain contributions like that of New Zealand to the war in Afghanistan; states that are unaffected by the outcome of the conflict, that have no notable ability to influence the outcome of the war, that experience no reputational cost from refusing to participate, and yet willingly risk high costs from their participation.
	
	New Zealand is not an isolated incident. Table \ref{table:2001_top} shows that many of the states that contributed the highest share of their armed forces to ISAF were not those that current theories would anticipate. Of the 10 states that committed the largest share of their military forces to the theater at the war's start, only Turkey is geographically proximate and only 5 of the 9 non-US countries were NATO members.\footnote{Canada's conflict-leading contributions are omitted because this data only references 2001.} And although every NATO member participated in at least some capacity, over a half dozen had no physical presence in Afghanistan until more than halfway through the mission.

	\begin{table}[ht]
		\centering
		\begin{tabular}{|lr|}
			\hline
			\textbf{State} & \textbf{Contribution} \\
			\hline
			Denmark & 0.66 \\
			New Zealand & 0.65 \\
			United States & 0.53 \\
			Romania & 0.53 \\
			Netherlands & 0.44 \\
			Germany & 0.44 \\
			Norway & 0.34 \\
			Australia & 0.29 \\
			Turkey & 0.27 \\
			Spain & 0.20 \\
			\hline
		\end{tabular}
	\caption{Top 10 contributors to ISAF (2001) by percent of their armed forces. Source: IISS Military Balance (2002)}
	\label{table:2001_top}
	\end{table}

	This paper explains such contributions by developing a new theory about coalition warfare participation by states whose primary objective in fighting is not to influence the outcome of the conflict or fulfill the role expected of close and reliable allies. Instead, states contribute forces to coalition warfare in order to \textit{become} close allies with a reputation of reliability. These states make costly contributions not because they already have close relationships with the coalition leader, but because their relationship is not close enough. As such, their contribution is disproportionately costly to signal their desire for a stronger relationship with their coalition partner. We find that if state military contributions are measured not by the number of troops they contributed, but the proportion of their military forces they contributed, that states seeking to develop a stronger relation with coalition leaders are more likely to have higher proportional troop contributions. In other words, states that wish to increase the strength of their relationship to central actors in the coalition seek to do so by over-contributing, sending a larger portion of their armed forces than states whose participation was formally obligated or otherwise expected.
	
	We test our theory using new data on each country's relative contribution to the Afghanistan war on an annual basis as well as each country's latent relationship strength with the United States. Unlike prior analyses that focus on raw troop count, we consider the cost of each state's troop contribution as a ratio of the size of their armed forces. If a small state is trying to signal to the coalition leader that it is willing to overexert itself on behalf of that coalition leader, then its relative contributions matter most. Rather than predicting a state's participation in war based on the nature of their current relationship, looking at the discrepancy between the closeness of their current relationship and the relationship one would expect given foreign policy interests can help identify the group of states that, like New Zealand, may over-contribute in order to improve the closeness of the current relationship. Our analysis confirms that states whose security relationship with the US is weaker than it should be, given the alignment of their respective security interests, contributed a larger share of their armed forces to the US-led war in Afghanistan (ISAF).
	
	This finding has important implications for understanding the costly means by which states seek to re-align themselves in international alliance networks as well as how states fight. Conflict is generally understood as a costly tool which states employ to achieve their international objectives. One of those objectives can be unrelated to the outcome of the conflict itself. States may use war efforts as a of their willingness to undergo large costs in the hopes that this will improve their relationship with the coalition leader in the future. This can help explain the ways that states use conflicts which are not of immediate strategic importance to -- hopefully -- gain the attention and respect of important actors in the international system. Simply put, war can be a good excuse for improving ties with like-minded states. States are willing to incur costs in an institutional context not when they are already well-embedded in those institutions, but when they have the most to gain from further institutionalization. Our findings point to the importance of understanding who joins military coalitions and why \citep[12-14]{wolford_politicsmilitarycoalitions_2015}. The quality of coalitions impacts factors like military skill, coordination, and legitimacy \citep{auerswald_natoafghanistanfighting_2014, saideman_ambivalentcoalitiondoing_2016, cranmer_coalitionqualitymultinational_2017, cappellazielinski_dictatorsfightingtogether_2018}. But having more eager partners is not always an obvious gain. Receiving coalitional support from acquaintances rather than close allies could reduce the ease of coordination and increase the cost of side payments \citep{papayoanou_intraalliancebargainingbosnia_1997, morrow_alliancesasymmetryalternative_1991, wolford_politicsmilitarycoalitions_2015}. During the Korean War, for example, the United States turned down Pakistan's offer of coalition forces due to fears of operational efficiency and a loss of US control over the conflict \citep{stueck_koreanwarinternational_1997}. Unanticipated defection by partners that initially joined Operation Iraqi Freedom similarly increased expenses for those that remained \citep[12-13]{mcinnis_varietiesdefectionstrategies_2018}.

	This paper proceeds in five parts. In part two we explore existing explanations for coalition warfare, which thus far have focused on the decision to join coalition efforts rather than the extent of their participation. Part three develops our theory by applying the costly signaling theory to coalition warfare through a novel measure of the costliness of a state's contribution to coalition warfare: the relative pressure of that mobilization based on the size of its available armed forces. Part four empirically test this finding by examining coalition contributions during the war in Afghanistan (2001-2014), which presents a ripe test case for our theory given variation in the alliance obligations of the states that participated as well as their level of participation. Section five includes a discussion of the generalizability of our results for the broader theory, explaining how states seek to alter and leverage their position within the network of capable international actors. Section six concludes.

\section{Existing Theories of Coalition Warfare Contributions}
	An alliance between two actors constitutes a promise to take a certain action in the event of a particular contingency \citep[526]{altfeld_decisionallytheory_1984}. In the international security context, an alliance is thus often a promise to defend the other actor in the event of a threat to their security \citep{waltz_theoryinternationalpolitics_1979, walt_originsalliance_1987} or to facilitate mutually beneficial cooperation when states have common objectives \citep{keohane_hegemonycooperationdiscord_1984, wolford_showing_2014}. Although the importance of theorizing differing contributions has been recognized in the context of US-led coalition efforts, this work has focused on single-dyad case studies that would benefit from broader analysis we provide that confirms the generalizability of their findings \citep[4-5]{mello_politicsmultinationalmilitary_2018}. Attempts to explain why countries contribute to alliances in varying degrees can be divided into four different categories; theories of collective action, the balance of threat, alliance dependence, and domestic politics \citep{bennett_burdensharingpersiangulf_1994, haesebrouck_democraticparticipationair_2016}.
	
	The \textit{collective action hypothesis} introduced by \citet{olson_economictheoryalliances_1966} argued that dominant states will end up making the largest contributions to alliances because smaller states can simply free ride while continuing to garner the benefits of the alliance relationship writ large. These dominant states, with larger economies and militaries, end up paying a disproportionate burden to secure goods even when the public benefits of those goods are reaped by states that made little contribution themselves. The \textit{balance of threat hypothesis} argues that state contributions should be proportional to the gravity of the threat an issue presents to a state. When faced with a larger threat, a state will contribute more to an alliance coalition that they anticipate mitigating that threat \citep{walt_originsalliance_1987, baltrusaitis_coalitionpoliticsiraq_2010, sandler_natoburdensharing_2014}. The \textit{alliance dependence hypothesis} argues that states in an alliance must balance two competing fears; the fear of abandonment and the fear of entrapment \citep{snyder_securitydilemmaalliance_1984}. Problematically, reducing one of these risks necessitates an increase in the risk of the other. Allied support is thus explained by how a state feels about these relative risks; states will contribute to an alliance when the fear of abandonment exceeds the fear of entrapment. States that most fear abandonment will be those that are most dependent on the other partner in the alliance meaning the military and economic payoffs of the alliance relationship would be difficult to replace. This theory has been expanded by scholars who note that ``alliance value" explains contributions by states who believe they can leverage their allies in their favor \citep{davidson_neoclassicalrealistexplanation_2011}. The \textit{domestic hypotheses} of alliance contributions take multiple forms \citep{,tago_whenaredemocratic_2009, ashraf_politicscoalitionburdensharing_2011, pilster_aredemocraciesbetter_2011, wolford_nationalleaderspolitical_2016}. Theories of state autonomy and domestic society predict that leaders that can insulate themselves from external constraints on their decision-making about alliance contributions are consequently able to do so even when these leaders' preferences regarding contributions differ from those of the public \citep{saideman_ambivalentcoalitiondoing_2016, vonhlatky_ideologyballotsalliances_2018}. Theories of bureaucratic politics instead examine the relationship within the government. Bureaucratic decision-making requires negotiations and bargaining among relevant actors and as such state contributions are the outcome of these bargaining decisions and the environment shaping the bargaining framework \citep{rathbun_partisaninterventionseuropean_2004, mello_democraticparticipationarmed_2014}. 

	Other literature on state contributions to joint efforts have examined topics like UN peacekeeping. In this context, state contributions are influenced by network centrality \citep{dorussen_networkedinternationalpolitics_2016}, international factors \citep{mullenbach_decidingkeeppeace_2005}, geostrategic interests \citep{baltrusaitis_friendsindeedcoalition_2008}, and regime type \citep{lebovic_unitingpeacedemocracies_2004}. While theoretically important, these findings do not explain the costliness of a state's contribution nor whether that contribution was used as a signal to improve a state's ties with other actors in the network. Their network position is assumed as a static factor explaining the level of contribution when in reality it is a position that states want to actively manipulate. The costliness of UN peacekeeping contributions as a signal is harder to measure considering the risk of casualties and collateral damage is lower and domestic publics are less attuned to their state's participation in UN peacekeeping than active military operations. 

\section{A Theory of Joint Coalition Warfare as a Costly Signal}
	Why do states cooperate with one another on security issues and how do states shore up and maintain that cooperation? Though there is an extensive literature on the ways in which states aggregate their capabilities to bolster their security against foreign threats \citep{waltz_theoryinternationalpolitics_1979}, the security benefits of an alliance do not precisely map onto the security benefits of joining a wartime coalition because coalitions and alliances are not synonymous. Coalitions can be a manifestation of an alliance promise or be an ad hoc relationship oriented toward an immediate goal rather than a broader security arrangement \citep[115]{weitsman_wartimealliancescoalition_2010}. The benefits of the alliance can be reaped even if a participant in the alliance free rides \citep{olson_economictheoryalliances_1966}. During wartime coalitions, however, there are more clear private benefits to states from participating in that coalition. Advocates of the balance of threat perspective point to private incentives actors have to ensure that the aggregate contributions to defense are sufficient to deal with the threat \citep{bennett_friendsneedburden_1997, baltrusaitis_coalitionpoliticsiraq_2010, davidson_neoclassicalrealistexplanation_2011}.
	
	Our theory thus lies at the intersection of research on coalitions and alliances to explain why states contribute more or less to coalitions than we would expect \citep{saideman_ambivalentcoalitiondoing_2016}. We argue that coalitions are used as a costly signal by a state that is dissatisfied with a more general alliance relationship they regard as too weak. States get private good benefits from coalition war-fighting. For some states, that private good is a desirable war outcome. For others, the private good to be gained from coalition war-fighting is improving the quality of your relational tie with the central actors leading that conflict. This tie is most improved when a state's contribution to the coalition effort is a costly signal of their commitment to the conflict which happens when they have over-contributed relative to the contribution that would be expected from a state with their military capacity. Since alliances are a costly signal of one's intention to cooperate, it should also be true that fulfilling alliance-like obligations even when you are not bound by those alliance obligations would also be a signal of your desire to cooperate more in the future \citep[704]{warren_geometrysecuritymodeling_2010}. Participation in war efforts by fellow alliance members is one way states can signal that they are not free riding on the alliance membership and are instead participating in burden sharing because they value the reciprocal and mutually-beneficial alliance commitments \citep[225-227]{maskaliunaite_sharingburdenassessing_2014}.

	We argue that national defense is a club good, not a pure public good, because the protection benefits of an alliance can be excludable and rival \citep[336]{sandler_clubtheorythirty_1997}. After the end of the Cold War, NATO's security benefits became something the US would leverage via threats of exclusion to encourage risk-sharing and minimize free riding \citep[324-325]{ringsmose_natoburdensharingredux_2010}. As a result, states within security alliances must avoid the temptation to free ride because doing so risks losing the benefits of alliance membership. States thus have to devote resources to maintain the receipt of these club goods and can do so by demonstrating what they bring to the table. For example, President Trump's rhetoric regarding NATO burden-sharing has resulted in conversations within Canada about how to avoid the perception that they are free-riding on US security commitments \citep[143]{mckay_whycanadabest_2018}.
	
	This private goods theory of coalition conflict, as distinct from alliances, explains why states would vary in the \textit{degree} of their contribution. Acquaintances make a costly contribution when they could have chosen to free ride because their interest is not in the public good of security achieved via victory in the conflict but private interests like a security umbrella down the road or closer economic or diplomatic ties with the state to whom they sent a costly signal of support. Previous literature has noted that you should get a payoff from signaling that you honored an alliance commitment and states that do so get more allies in the future because they are seen as reliable \citep[427-428]{gibler_costsrenegingreputation_2008}. This is consistent with our logic where you hope that your signal of strongly valuing the relationship causes the other actor to later take actions that recognize that.
	
	\subsection{New Conceptions of Contribution Cost and Dynamic Interstate Alliances}
		We depart from current explanations for state contributions to coalition war efforts in how we conceptualize and empirically identify that contribution. The extent to which a country participates in a coalition war effort is distinct from whether they participate at all. The foundational literature on theories of collective action, the balance of threat, and alliance dependence typically examine whether a country participated; only domestic explanations focus on constraints regarding the \textit{extent} of contribution \citep{bennett_burdensharingpersiangulf_1994, bogers_missionafghanistanwho_2013}.

		\begin{figure}
			\centering
			\includegraphics[scale=0.85]{figures/contr_map_both.pdf}
			\caption{Country troop contributions to war in Afghanistan (2001-2014). The top figure displays the percent of total ISAF troops contributed by each country. The bottom figure displays the percent of a country's total troops contributed. By viewing contributions in terms of a country's available troops, rather than the total number of troops in the \emph{coalition}, countries like New Zealand and Denmark are revealed as committed contributors. Source: IISS Military Balance (2002 -- 2015)}
			\label{fig:contrib_map}
		\end{figure}
	
		Contrasting findings among the first three theories are explained by examining the extent of state contributions rather than just whether they contributed \citep[4]{cranmer_coalitionqualitymultinational_2017}. Yet previous work examining the extent of state contributions suffers from improper conceptualization. Almost all prior analysis has measured a state's contribution in absolute terms; the number of troops, financial contributions, foreign aid, or peacekeepers sent to a particular operation \citep{mello_democraticparticipationarmed_2014, haesebrouck_explainingmemberstates_2016}. This measure makes sense for theories interested in explaining who contributes the most, since absolute terms describe the highest contributors \citep[40-41]{bogers_missionafghanistanwho_2013}. But absolute contributions do not tell us who contributes more or less than expected nor does it tell us the relative cost of that contribution for the state in question. It is less costly for Great Britain to send 1,000 forces into a conflict theater than for Poland to do the same since Poland's comparatively small military force makes that mobilization an unusually large undertaking. The former has a substantially larger military force and thus experiences a lower national cost in terms of the burden such a contribution places on its military forces and the zero-sum trade-off of this contribution relative to other security or financial needs. Figure \ref{fig:contrib_map} shows that a measure of a state's absolute contribution (here the percent of \emph{all} troops that are deployed in Afghanistan) provides different rankings of costliness than the percent of a state's relative contribution (the percent of \textit{their} troops that are deployed in Afghanistan). While the US is the largest contributor by both measures, looking at contributions as the percent of a state's armed forces demonstrates that countries like Denmark, New Zealand, Romania, and Australia were devoted to the conflict to a surprising extent. Measuring troops contributed as a ratio of other baselines like population size or GDP yield similar results \citep[41]{bogers_missionafghanistanwho_2013}.

		The second problem is how current theories think about alliances. Conventional explanations anticipate the highest contributions coming from states that are dependent on the security guarantee of the central coalition actor and those that believe they have a special relationship with the United States \citep{graeger_revivalatlanticismnato_2009, biehl_strategiccultureseurope_2013, howorth_securitydefencepolicy_2014, haesebrouck_democraticparticipationair_2016}. However, this views the alliance relationship as static and oriented towards maintaining continuity in the current relationship rather than being forward-looking and seeking change. As such, the assumption is that countries that fight together do so because of their closely aligned interests. This misses how fighting together can be a method for \textit{altering} the perceived alignment of interests, not maintaining them. As we demonstrate, state contributions are positively associated with a state's opportunity to subsequently strengthen its alliance with the US. By understanding this dynamic nature of alliances, important strategic considerations emerge.
		
	\subsection{Benefits of Over-contributing Forces}
		Our theory posits that countries will make the costliest contributions not when they are already closely aligned with the coalition's leader, but when the contribution can be used to improve an under-performing alliance relationship. The most costly contributions do not come from those with the strongest ties to the coalition leader \citep{ringsmose_natoburdensharingredux_2010, wolford_showing_2014}, those who care about their broader international reputation \citep{pedersen_bandwagonstatuschanging_2018}, or those who are capitulating to demands from the coalition leader \citep{schweller_newrealistresearch_1997}. Rather, over-contributing is a way to send a separating signal that even though a state is not closely aligned with the central actor prior to or at the outset of the conflict, it is willing to incur costs for the sake of the central actor in the hopes that it improves their alignment.

		Costly over-contributions serve as an alliance signaling function under two conditions. First, the over-contributing state must be dissatisfied with the current alliance relationship. While the conventional alliance hypotheses argue that the highest contributions will come from states that have a special relationship with the United States or that are currently dependent on the security guarantee, we argue instead that a different set of states -- those that want a more tightly linked relationship with the United States -- will make more costly contributions. States that already have a special relationship or a reliable security guarantee do not need to over-contribute because their relationship is, in a sense, locked-in. These states have some flexibility in their expected contributions and are unlikely to lose their special relationship status simply because they did not go above and beyond in contributing to the coalition effort. They just need to contribute enough that they are not seen as free ridering. 

		We use a latent measure of alliance depth in order to identify the current depth of a state's alliance with the United States. By identifying this latent measure, we are able
		to differentiate predictable alliance ties from those that are unanticipated. Given the close similarity in foreign policy preferences, the deep alliance relationship between the US and the UK is not surprising. The non-existence of any US-Venezuela security alliance is similarly predictable if a hunch was made based on the non-alignment of those states' foreign policy preferences. We would not expect either the UK or Venezuela to be dissatisfied with the current state of their security alliance with the US even though one represents a close security alliance and the other a non-existent one. Our focus is thus states like Australia that don't fit this category. Australia -- despite historically consistent interests with the US regarding intelligence and security threats -- has a weak, informal, and largely ad-hoc security relationship with the United States \citep{fruhling_anzusreallyalliance_2018}.
		
		Importantly, our theory is not necessarily inconsistent with theories of domestic politics. domestic factors could be the reason a state wants to establish itself as a closer ally with a powerful country \citep{tago_whenaredemocratic_2009, pilster_aredemocraciesbetter_2011, wolford_nationalleaderspolitical_2016}. The domestic origins of alliance-motivated coalition contributions have been witnessed in Egypt in the 1960's \citep{barnett_domesticsourcesalliances_1991}, the United Kingdom during the Iraq War \citep{davidson_americaallieswar_2011}, and Canada since the war in Afghanistan \citep{massie_alliancevaluestatus_2018, mckay_whycanadabest_2018}. In these cases, political leaders who fear domestic opposition from other government actors like bureaucrats or non-partisans or from non-government actors like voters or interest groups may use alliances with other states to demonstrate competent policymaking. The reasoning behind a state's motivation for signaling their interest in a closer interstate relationship represents an important area of future research.
		
		Second, the leading state must perceive the contributing state's effort as occurring for the leading state's benefit. For an over-contribution to be a costly signal of one's commitment to their ally, the private incentives to contribute cannot be egoistic security \citep{davidson_americaallieswar_2011}. If the contributing partner garners substantial private security benefits from participating in the war effort, the central coalition actor would be less likely to interpret the over-contribution as a signal of good will and would instead interpret it as a self-interested intervention that would have occurred irrespective of its benefits to the coalition leader \citep{tago_whystatesjoin_2007, lake_hierarchyinternationalrelations_2009, chapman_securingapprovaldomestic_2011}. Because the value of the signal is determined by its cost, not by its effect, it does not matter if the signaling state sends forces that can materially influence the outcome of the conflict \citep{davidson_americaallieswar_2011}. It only matters that the signal is interpreted as one that was costly for the sender irrespective of its impact on the war's outcome. In most cases, the powerful state not only does not need support from smaller states in order to win the conflict, but may actually want to maintain control of the forces that will determine the conflict outcome. This is consistent with alliance theories about the autonomy benefits one actor gets from asymmetric alliances \citep{morrow_alliancesasymmetryalternative_1991}. As such, smaller states do not want to make contributions that are too influential in the conflict's outcome because of the risk of intruding upon the powerful state's decision-making. Importantly, this differs from other perspectives like \citet[72-75]{bennett_burdensharingpersiangulf_1994} since the US is not exercising leverage over smaller states to induce their contributions; instead the momentum comes from the smaller initiative-taking states whose contribution does more to serve their long-term interests than the immediate interests of the powerful state in the conflict theater.

		The cost of partner contributions to coalition warfare is salient in the minds of policymakers and noticed by coalition leaders \citep[328]{ringsmose_natoburdensharingredux_2010}. US Secretary of Defense Gates noted during the war in Afghanistan that there was a ``two-tiered alliance" that made a distinction between ``allies willing to fight and die to protect people's security and others who are not". This has been observed with Nordic states who make costly contributions to improve their relationship with the US because that relationship improves their international status \citep{pedersen_bandwagonstatuschanging_2018}. Our theory differs importantly from the ``alliance value" explanation provided by \citet{davidson_americaallieswar_2011} and \citet{massie_democraticalliesfollowership_2016} who hypothesize about the importance of present alliance value. We argue instead that states with a strong but stable current alliance have fewer incentive to over-contribute because they don't get any additional payoff from doing so and there is little risk of fallout from a normal or small contribution because the alliance is already strong \citep{davidson_headingexitsdemocratic_2014}. This is why small states are able to free ride and take the relationship for granted as pointed out by \citet{keohane_biginfluencesmall_1971}, but we don't always see this because states that desired an improved tie will in fact try to signal that they are not free riders. Canada, for example, was concerned about accusations of free riding on the US alliance and thus intensified their participation in Afghanistan to elevate their status in the eyes of American policymakers \citep{massie_alliancevaluestatus_2018}. If the US-Canada relationship had already been sufficiently strong, there would have been a reduced need to signal their value to the relationship. Rather, acquaintances that do benefit from signaling a change in their desired alignment tie will undertake more intense participation \citep{gibler_priorcommitmentscompatible_2004, gartzke_contractsfriendsalliances_2012}.

		\begin{hyp}
			A state with an interest in improving its security alliance with a coalition leader will make more costly contributions to that state's coalition conflicts than a state that is satisfied with its current security alliance.
		\end{hyp}

		Our theory, presented in figure \ref{fig:theory}, thus predicts a positive relationship between the desired strength of a state's alliance with the central coalition actor and that state's expected cost of contributing to the coalition war effort. The allies already optimally aligned with the central coalition actor may make the highest absolute contributions by virtue of their size and military strength, but they are not the states that make the highest relative contributions since they have no incentive to incur the costs that high relative contributions entail. Instead, it's those that want to be closer allies, relatively speaking, that contribute the most.

		\begin{figure}[H]
		\begin{subfigure}{.5\textwidth}
			\begin{tikzpicture}
				% Graph axes
				\draw[thick,<->] (0,5) node[above]{\begin{tabular}{c}
					Contribution \\ \textbf{Amount}
					\end{tabular}} -- (0,0) -- (5,0) node[below]{\begin{tabular}{c}
					\textbf{Current} Alliance \\ Strength
					\end{tabular}};
	
				% Contribution Curve
				\draw[->] (0.2,0.2) -- (4,3);
			\end{tikzpicture}
			\caption{Active Alliance Strength Theory}
			\label{fig:theory_old}
		\end{subfigure}
		\begin{subfigure}{.5\textwidth}
		\begin{tikzpicture}
		% Graph axes
			\draw[thick,<->] (0,5) node[above]{\begin{tabular}{c}
				Contribution \\ \textbf{Cost}
				\end{tabular}} -- (0,0) -- (5,0) node[below]{\begin{tabular}{c}
				\textbf{Desired} Alliance \\ Strength
				\end{tabular}};
		
		% Contribution Curve
		\draw[->] (0.2,0.2) -- (4,3);
		\end{tikzpicture}
		\caption{Latent Alliance Signaling Theory}
		\label{fig:theory_new}
		\end{subfigure}
		\caption{Theory of State Contributions to Wartime Coalitions}
		\label{fig:theory}
		\end{figure}

\section{Research Design}
	\subsection{Data}
		
		Our dependent variable is the extent of a state's contribution to the war in Afghanistan, which we measure using data on country-year troop contributions. Rather than relying on a raw measure of the number of troops a country sent to Afghanistan, we measure the cost of a contribution as the share of a state's armed forces that they deployed to Afghanistan in a given year. By measuring a state's contribution as the \emph{percent} of its total troops rather than the \emph{number} of troops we are able to measure the costliness of a state's contribution in a way that discriminates between states. 
		
		Data on each country's annual troop contributions and their overall military personnel were collected from the International Institute for Strategic Studies (IISS) Military Balance reports from 2002 -- 2015 \citep{iiss}.\footnote{The reports publish data on the prior year's activities, so all annual reports have been lagged to reflect the calendar year they describe rather than the year of publication.} Military personnel and deployment data from IISS has been used widely by previous scholars \citep[e.g.][]{walter_buildingreputationwhy_2006, rovner_hegemonyforceposture_2014, beckley_emergingmilitarybalance_2017, henke_politicsdiplomacyhow_2017} and unlike other sources used for the war in Afghanistan, benefits from coverage over the course of the entire war. Figure \ref{fig:afghan_total} summarizes troop numbers in Afghanistan for year and figure \ref{fig:troop_hist} shows the distribution of all country-year contributions. Some states like Denmark, New Zealand, the United States and Romania contributed roughly 3 times the size of their armed forces than did the median contributor. 

		\begin{figure}[H]
			\centering
			\includegraphics[width = \textwidth]{figures/troop_year.pdf}
			\caption{Annual distribution of foreign troops in Afghanistan (Source: IISS Military Balance Report)}
			\label{fig:afghan_total}
		\end{figure}
		
		There are a variety of ways to measure burden sharing, including the number of troops as a share of population size, armed forces size, armed forces in the coalition, and GDP \citep[668-669]{hartley_natoburdensharingfuture_1999}. The costliness of a state's contribution is measured as a share of the size of their armed forces as opposed to GDP or military spending since the risk of casualties and collateral damage are two costs unique to personnel contributions that states are attentive to when deciding whether and to what extent they should participate in coalition warfare \citep{ringsmose_natoburdensharingredux_2010, chivvis_topplingqaddafilibya_2014, vonhlatky_cashcombatamerica_2015, haesebrouck_natoburdensharing_2017}. This measure thus best captures the security burden a country has willingly undertaken to fight in another state's war. We do not distinguish between the type of troop or whether they were in an active combat zone because of the unreliable or classified nature of that data \citep[44-45]{bogers_missionafghanistanwho_2013}. Furthermore, while military tacticians may interpret troops sent to a combat zone as a more costly contribution than those sent elsewhere, national politicians -- the intended recipients of that signal -- are less likely to meaningfully make that distinction.

		Our theory expects that contributions will be positively associated with the amount that a state wants to improve its alliance with the United States. We calculate a state's desired increase in alliance strength through two variables. A state's ideal alliance strength is measured by its preference similarity with the United States -- the similarity of its United Nations voting patters to those of the United States. We operationalize the current strength of a state's alliance by replicating and extending Benson and Clinton's (\citeyear{benson_assessingvariationformal_2016}) measure of alliance depth, which we then standardize to range from zero to one.\footnote{Benson and Clinton produce measures of alliance depth and strength through 2000 in their paper. Thanks to the recently released update to the ATOP data \citep{leeds_alliancetreatyobligations_2002} which now includes alliance information up through 2016, we are able to use the same models from Benson and Clinton's original paper  but with an extended temporal scope.} Benson and Clinton's framework improves upon binary measures of alliance presence by instead allowing us to not only capture the presence of an alliance, but also a continuous measure of an alliance's overall value.

		We then estimate the alliance strength two states should have, given how closely aligned their security interests seem to be, through UN voting data. This is consistent with operationalizations used in previous work on coalition warfare \citep{wolford_politicsmilitarycoalitions_2015}. We assume that states with similar security interests should have a formal alliance relationship that reflects this alignment. Of course, that is not always the case. Our ``potential alliance gain" variable is thus the difference between ideal alliance strength and actual alliance strength. We predict that a state whose interests are very closely aligned with the US but that has only a middling relationship with the US should behave as current theories predict closely aligned state will behave (i.e. they will make a costly contribution to the coalition war effort). They will do so in order to signal their desire for a stronger relationship that better represents their true alignment of interests than the current alliance relationship. The earlier example comparing the UK, Venezuela, and Australia is illustrative of this concept. Australia represents an example of a state with a high latent relationship -- meaning an alliance relationship that should be stronger than what one observes. We predict that states that have a weaker relationship with the US than they desire are more likely to choose costly contributions to signal that dissatisfaction to the United States in the hopes that such a signal will curry favor and result in an improved relationship. States that are satisfied with their current relationship with the US (e.g. UK and Venezuela) are less likely to make unexpectedly costly contributions regardless of the strength of their current relationship.

		\begin{figure}[ht]
			\centering
			\includegraphics[width=0.65\textwidth]{figures/troop_dist.pdf}
			\caption{Distribution of Country Contributions to Afghan war (2001-2013) Source: IISS Military Balance report}
			\label{fig:troop_hist}
		\end{figure}
		
	\subsection{Model and Results}
		Our unit of analysis for these models is the country-year, spanning from 2001-2014. Our dependent variable for each model is the percent of a state's total troops contributed to Afghanistan in a given year. Because our theory is linear and our outcome is continuous we estimate a series of linear regressions. We also estimate a fixed effects model to check if the estimated effect of interest holds once time-invariant unobserved confounders are accounted for. The primary predictor of interest is the difference between a country's United Nations voting similarity with the United States and the depth of its alliances with the United States -- its potential for growing the alliance closer to its desired alignment.
		
		We include a series of relevant controls: whether a country is a democracy \citep{gartzke_whydemocraciesmay_2004}, its distance from the U.S. by a popular ideal point measure \citep{bailey_estimatingdynamicstate_2017}, the number of casualties which a state has incurred throughout the conflict, whether or not the country has any alliance with the United States, a country's CINC score, GDP (logged), GDP per capita (logged), and geographic distance from the conflict theater.
		
		\begin{table}[H]
\begin{center}
\begin{tabular}{l c c c c }
\hline
 & Baseline & Political Controls & All Controls & Fixed Effects \\
\hline
Intercept      & $0.000^{*}$       & $0.002^{*}$         & $0.007^{*}$         &                     \\
               & $[0.000;\ 0.001]$ & $[0.001;\ 0.002]$   & $[0.002;\ 0.012]$   &                     \\
Potential Gain & $0.012^{*}$       & $0.014^{*}$         & $0.009^{*}$         & $0.010^{*}$         \\
               & $[0.011;\ 0.013]$ & $[0.012;\ 0.016]$   & $[0.006;\ 0.012]$   & $[0.003;\ 0.016]$   \\
Ideal Point    &                   & $0.002^{*}$         & $0.001^{*}$         & $-0.002^{*}$        \\
               &                   & $[0.001;\ 0.002]$   & $[0.000;\ 0.002]$   & $[-0.003;\ -0.000]$ \\
Democracy      &                   & $0.000$             & $0.000$             & $0.001$             \\
               &                   & $[-0.000;\ 0.001]$  & $[-0.001;\ 0.001]$  & $[-0.000;\ 0.002]$  \\
Casualties     &                   & $0.000^{*}$         & $0.000^{*}$         & $0.000^{*}$         \\
               &                   & $[0.000;\ 0.000]$   & $[0.000;\ 0.000]$   & $[0.000;\ 0.000]$   \\
U.S. Ally      &                   & $-0.003^{*}$        & $-0.001$            & $0.006^{*}$         \\
               &                   & $[-0.004;\ -0.002]$ & $[-0.002;\ 0.001]$  & $[0.001;\ 0.011]$   \\
CINC Score     &                   & $-0.020^{*}$        & $0.006$             & $-0.067$            \\
               &                   & $[-0.034;\ -0.006]$ & $[-0.011;\ 0.022]$  & $[-0.177;\ 0.042]$  \\
Log GDP        &                   &                     & $-0.001^{*}$        &                     \\
               &                   &                     & $[-0.001;\ -0.000]$ &                     \\
Log GDPPC      &                   &                     & $0.001^{*}$         &                     \\
               &                   &                     & $[0.001;\ 0.001]$   &                     \\
Distance       &                   &                     & $-0.000^{*}$        &                     \\
               &                   &                     & $[-0.000;\ -0.000]$ &                     \\
\hline
AIC            & -17677.705        & -13883.175          & -6404.655           &                     \\
BIC            & -17660.471        & -13839.153          & -6353.043           &                     \\
Log Likelihood & 8841.852          & 6949.587            & 3213.328            &                     \\
Deviance       & 0.064             & 0.050               & 0.016               &                     \\
Num. obs.      & 2309              & 1813                & 806                 & 1813                \\
\hline
\multicolumn{5}{l}{\scriptsize{$^*$ 0 outside the confidence interval}}
\end{tabular}
\caption{Statistical models}
\label{table:coefficients}
\end{center}
\end{table}


		As Table 2 demonstrates, regardless of model specification we find that the coefficient for potential gain with the United States is positive and significant. Moreover, when these estimates are considered in terms of predicted contributions, then we see that a substantial amount of variation occurs. For this figure we hold all predictors at their central values and then calculate the predicted contribution at each possible value of potential alliance gain with the United States. For the median military size of 26,500 troops -- which is arguably more representative of most countries because the distribution of military size is heavily right-skewed -- the range of predicted contributions spans from 39 to 414 troops. For the mean military size -- 111,976 troops -- the difference between the maximum and minimum predicted contribution is 1,585 troops. If one considers, the political costs of sending troops abroad, then it becomes apparent that these coefficient estimates are not just statistically significant, but substantively meaningful.
	
		\begin{figure}[ht]
			\centering
			\includegraphics[scale = 0.75]{figures/pred_contr.pdf}
			\caption{Predicted contributions by potential for gain in U.S. alliance depth. The x-axis is the range of potential gain in alliance depth with the United States. The y-axis is the predicted troop contribution as a percent of a country's troops. For the median military -- 26,500 troops -- predicted contributions range from 39 to 414, a difference of 375 troops.}
		\end{figure}

\section{Implications}
	Because alliance membership is a club good, not a pure public good, worries about being excluded from the security benefits of an alliance can incentivize a state to try to shore up the alliance by sending a costly signal of their commitment to their alliance partners. ``Today, we should expect the European allies to find that the best way to strengthen (or avoid weakening) their bonds with the United States is to contribute to out-of-area operations like ISAF. According to this line of reasoning, the allies who put the highest premium on NATO’s traditional products should be the ones – together with the United States – shouldering the heaviest burdens in Afghanistan" \citep[331]{ringsmose_natoburdensharingredux_2010}. Yet the Afghanistan case demonstrates this is not just a story about NATO membership, but security cooperation more generally.

	Our findings run counter to expectations of theories about the balance of threats \citep{walt_originsalliance_1987}. For the balance of threat hypothesis, the states with the highest contributions are those that are most concerned about the threat the opposing state presents to their security \citep{haesebrouck_democraticparticipationair_2016}. Yet, when contributions are measured as internal costs relative to the contribution a state could have made, the highest contributions are from states that appear to be opting into a war in which they have no material stake. Denmark, New Zealand, and Romania, 3 of the top 4 contributors to ISAF in 2001 by our measure, are not the states that are most concerned about the threat that the Taliban and Al-Qaeda posed to their national security. This empirical observation is not explained by the theory that states enter an alliance in order to balance against threats.

	Previous theories described under the rubric of the alliance hypothesis also fail to explain this finding. For previous theories, ally support is expected when you can leverage your allies' power in your favor and is thus positively associated with closeness \citep{davidson_neoclassicalrealistexplanation_2011}. This expects states that are the closest allies with the United States to be the largest contributors to US war efforts. Yet the relationship we see does not follow this trend. States do not over-contribute because the United States is their best protector or because they think they have the ability to influence the United States through their international clout as \citet{ringsmose_natoburdensharingredux_2010} finds. Rather, it is the states that want the United States to be their best protector or that \textit{want} the ability to influence international relations with the United States that are most likely to over-contribute \citep{vonhlatky_greatasymmetryamerica_2010}. It's also not the case that the US coerced its closest allies into participating as argued by \citet{kupchan_natopersiangulf_1988} because those are not the states that ended up doing the most participating relative to what they could have contributed. And while NATO membership and its accompanying obligations to avoid accusations of defection may be sufficient to explain why states do not free ride, that does not explain why states not bound by alliance obligations (eg. New Zealand) end up over-contributing. Thus, desire for stronger ties and the expectation that over-contribution will positively signal that desire explains our findings in a way that run counter to previous theories.

	While our analysis is limited to an examination of the war in Afghanistan, there is suggestive evidence that the theory holds for other coalition conflicts. During the war in Libya, Denmark made a concerted effort to over-contributed forces, particularly air forces to the bombing campaign, in order to demonstrate their ``relevance and trustworthiness to its great power allies in NATO, especially the United States" \citep[109]{jakobsen_goodnewslibya_2012}.\footnote{A detailed account of Denmark during the war in Libya is provided by \citet{dicke_natoburdensharinglibya_2013} and \citet{jakobsen_prestigeseekingsmallstates_2018}.} This effort appears to have paid off. US Secretary of Defense Robert Gates \citet{gates_securitydefenseagenda_2011} commended Denmark for its costly contribution to the conflict when he publicly noted that Denmark ``...with their constrained resources, found ways to do the training, buy the equipment, and field the platforms necessary to make a credible military contribution." Further research should thus investigate these recent conflicts once annual data about troop contributions is more reliably available.

\section{Conclusion}
	A state's closest allies will not be the states most invested in fighting alongside that state. Rather, weakly aligned states that desire a closer relationship are most likely to overcontribute forces to coalition warfare. This finding produces a different list of contributors than what current theories predict.
	
	The study of state contributions to coalition warfare provides unique insights if those contributions are measured relative to the cost they impose onto the contributor rather than the utility they provide regarding the conflict outcome. Not every state contributes to a war effort in order to influence the outcome of the war as theories of balance of threats predict. But it is also not the case, as predicted by the collective action hypothesis that states simply free ride when the conflict outcome is immaterial or, as predicted by the alliance dependence hypothesis, that the stronger your alliance, the more you contribute.

	Rather, because war is costly, it can serve as a signaling function for states that want to convey their desire to improve their relationship with other states in the international system. By accepting comparatively large costs to fighting alongside another state, especially when such fighting does not otherwise benefit you, states can signal how much they value an alliance relationship in a way that they anticipate will generate reciprocity and payoffs down the road. Future work should examine whether states are making a smart bet in anticipating future payoffs from costly over-contributions; it is possible their gamble is incorrect. But for now, it suffices to say that states with the most interest in future payoffs from a better alliance relationship are the same states most likely to separate themselves from the rest of the coalition pact by over-contributing to wartime coalitions in the hopes that doing so signals their reliability.

\bibliographystyle{apsr}
\bibliography{isaf_alliances}

\end{document}